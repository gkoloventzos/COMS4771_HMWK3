\section{Problem 1}
\subsection{}
In general there are 6 compinations that sum to 7 
\{\{1,6\},\{2,5\},\{3,4\},\{4,3\},\{5,2\},\{6,1\}\}. 
When the dice is fair all six of them has equal probability. 
So the probabilty to win with a fair dice is:
\begin{equation}
p_{fair-win} = \frac{6}{36} = \frac{1}{6}
\end{equation}
When the dice are unfair then the all of the probabilities 
change. So we now that 3 and for have $\frac{1}{3}$ probability.
So the rest is:
\begin{align*}
\frac{1}{3} + \frac{1}{3} + 4*x = 1 \\
\frac{2}{3} + 4*x =1 \\
x = \frac{1}{12}
\end{align*}
This is the probability of the rest of the dice {1,2,5,6}.
To find the probability to win with unfair dice we must 
calculate all the new probabilities. As we know that 
\{1,2,5,6\} is $\frac{1}{12}$ the probability for the 
\{\{1,6\},\{2,5\},\{5,2\},\{6,1\}\} is $\frac{1}{144}$ each.
So the probability to win with the unfair dice is 
\begin{align*}
p_{unfair-win} = 4*\frac{1}{144} + 2*\frac{1}{9} \\
p_{unfair-win} = \frac{1}{36} + \frac{2}{9} \\
\end{align*}
\begin{equation}
p_{unfair-win} = \frac{1}{4}
\end{equation}
Having calculated this now we are ready to start looking what 
the casino wants from us. 
The casino asks that we will create a function that will take
the number of rounds played (N) and outputs minimum number of 
wins that means that the dice are unfair.
This is equavalent to solving the following unequality:
\begin{equation} \label{eq:unequality}
P(Dice=Unfair\mid R=N,X=x) >0.5
\end{equation}
Where R is the random number of rounds and X is the number of wins.
From Bayes theorem the \ref{eq:unequality} is represented as
\begin{equation}
P(Dice=U\mid R=N,X=x) = \frac{P(R=N,X=x\mid D=U)P(D=U)}{P(X=x)}
\end{equation}
Using the law of total probability the denominator is changed as:
\begin{align*}
P(Dice=U\mid R=N,X=x) &= \frac{P(R=N,X=x\mid D=U)P(D=U)}{P(R=N,X=x\mid D=U)P(D=U)+P(R=N,X=x\mid D=F)P(D=F)}
\end{align*}
Because this is too big we are simplifing in
\begin{equation} \label{eq:unequality1}
P(Dice=U\mid R=N,X=x) =\cfrac{1}{1+ \frac{P(R=N,X=x\mid D=F)P(D=F)}{P(R=N,X=x\mid D=U)P(D=U)}}
\end{equation}
This probabilities are Bernoulli trials. So they obey to the binomial distribution.
This means that the probability of x wins after N rounds is:
\begin{equation} \label{eq:bin}
p(x,N) = \binom{N}{x} \cdot p^x(1-p)^{N-x}
\end{equation}
So the \ref{eq:unequality1} with \ref{eq:bin} becomes:
\begin{align*}
P(Dice=U\mid R=N,X=x) &=\cfrac{1}{1+ \frac{\binom{N}{x} \cdot p_f^x(1-p_f)^{N-x}P(D=F)}{\binom{N}{x} \cdot p_u^x(1-p_u)^{N-x}P(D=U)}} \\
P(Dice=U\mid R=N,X=x) &=\cfrac{1}{1+\frac{p_f^x(1-p_f)^{N-x}P(D=F)}{p_u^x(1-p_u)^{N-x}P(D=U)}}
\end{align*}
The problem gives us that $P(D=U) = \frac{1}{1000}$ so the $P(D=F) =999/100$
Having all this combined with the $p_f=1/6$ and $p_u=1/4$ we have:
\begin{align*}
\cfrac{1}{1 + (\frac{3}{5})^x\cdot(\frac{10}{9})^N\cdot999 } &> 1/2 \\
(\frac{3}{5})^x\cdot(\frac{10}{9})^N\cdot999 &<1
(\frac{3}{5})^x &< \frac{1}{999}\cdot(\frac{9}{10})^N
\end{align*}
As we need to solve for x we will take logarithm with base $3/5$
and we will change the direction of the unequaloty because the 
logarithm with base less than 1 is monotonically decreasing we have
\begin{equation}
x > N\cdot\log_{\frac{3}{5}}(\frac{9}{10})+\log_{\frac{3}{5}}(\frac{1}{999}))
\end{equation}
The minimum number of wins is:
\begin{equation} \label{eq:end}
x = \lceil 0.206N + 13.52  \rceil
\end{equation}
And this is the function the casino needs.
\subsection{}
Having the function \ref{eq:end} we need to observe at least
$\lceil 0.206N + 13.52  \rceil \geq N$ which we get $N = \lceil 17.02 \rceil = 18$
So for 18 rounds we get from the \ref{eq:end} function $x = \lceil 17.22 \rceil = 18$ wins.
So we need 18 consecutive wins to infere that the dice is unfair.
