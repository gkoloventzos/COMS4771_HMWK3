\section{Problem 3}
We will use a dice to give our counterexamples. So $\Omega = \{1,2,3,4,5,6\}$
\subsection{}
First we will show that $A\bot B \mid C$. We have as $A=\{1,4\}, B=\{2,4,5,6\}, C=\{4\}$.
$P(A\bigcap B\mid C) = \frac{P(A\bigcap B\bigcap C)}{P(C)} = 1$.\\
$\frac{P(A\bigcap C)}{P(C)} * \frac{P(B\bigcap C)}{P(C)} = 1$.\\
So our A,B,C are $A\bot B \mid C$.\\
Now we will show that $P(A\bigcap B) \neq P(A)*P{B}$.\\
$P(A\bigcap B) = \frac{1}{6}$ also $P(A) = \frac{1}{3}$ and $P(B)= \frac{2}{3}$.\\
So $P(A)*P{B} = \frac{2}{9} \neq \frac{1}{6} = P(A\bigcap B)$.\\
$A\bot B \mid C \not \rightarrow A\bot B$
\subsection{}
For this example we will have $A=\{1,4\}, B=\{4,5,6\}, C=\{1,5\}$.
As we can see C has one common with A and B respectively. But there are
no common digits all together. That means the $P(A\bigcap B\bigcap C) = 0$. But
$P(B\bigcap C)$ and $P(A\bigcap C)$ are not 0.
On the other hand $P(A\bigcap B) = \frac{1}{6}$ and $P(A) = \frac{1}{3}$ and $P(A) = \frac{2}{3}$.\\
So $P(A\bigcap B) = \frac{1}{6} = P(A)*P(B)$.\\
$A\bot B \not \rightarrow A\bot B \mid C$
\subsection{}
For this example we will have $A=\{1,2\}, B=\{2,3,4\}, C=\{4,5\}$.
In this example we have $P(A\bigcap B) = = \frac{1}{6} = \frac{1}{3} * \frac{1}{2} = P(A)*P(B)$.\\
Also $P(B\bigcap C) = = \frac{1}{6} = \frac{1}{3} * \frac{1}{2} = P(C)*P(B)$.\\
But $P(A\bigcap C) = 0 \neq P(A)*P(C)$.\\
$A\bot B \wedge  B\bot C \not \rightarrow A\bot C$
