\section{Problem 2}
\subsection{}
For $p(H)$ we know:
$p(H) = \int p(H,\mu) d\mu = \int p(H\mid\mu)p(\mu)d\mu = E\{p(H\mid\mu)\} = E\{\mu\}$
For A the $1$ is the pdf function as it is uniform.
For B we have that it will be of the form $ax^2+bx+c=y$
Also we know $f(0)=0$ which means $c=0$.
Also we know $f(1)=0$ which means $a=-b$.
And last $\int_0^1f(x)dx =1$ because it is probability.
\begin{align*}
\int_0^1ax^2-axdx &= 1 \\
\Big[ \frac{ax^3}{3} - \frac{ax^2}{2}\Big]_0^1 &=1 \\
\frac{a}{3} - \frac{a}{2} &= 1 \\
a &= -6
\end{align*}
So the pdf function is $-6\mu^2+6\mu=0$.
the $E\{\mu\} = \int_xp(x)f(x)dx$.
For our functions $f(x) = \mu$.
Giving all these for the A:
\begin{align*}
E\{\mu\} &= \int_{0}^{1} p(x)f(x) dx
=\int_{0}^{1}\mu d\mu \\
&=[\frac{m^2}{2}]_0^1
=\frac{1}{2} 
\end{align*}
For B:
\begin{align*}
E\{\mu\} &= \int_{0}^{1}p(x)f(x)dx
=-6\int_{0}^{1} (\mu^2-\mu)\mu d\mu \\
&=-6\Big[ \frac{\mu^4}{4}-\frac{\mu^3}{3}\Big]_0^1
=-6(\frac{1}{4}-\frac{1}{3}) \\
&=-6(-\frac{1}{12})
=\frac{1}{2}
\end{align*}
\subsection{}
To find the posterior distribution I have to calculate 4 numbers.
All of our experiments are subject to binomial distribution.
For function A:
In general the posterior distribution is found by:
$f(\mu|H=h,T=t) = \frac{Pr(H=h\mid \mu,N=h+t)g(\mu)}{\int_0^1 Pr(H=h\mid \mu',N=h+t)g(\mu') dr}$
But as the $Pr(H=h\mid \mu,N=h+t) = \binom{N}{h}\mu^h(1-\mu)^t$ and the $g(\mu) =1$ as uniform.
We have $f(\mu|H=h,T=t) = \frac{\binom{N}{h}\mu^h(1-\mu)^t}{\int_0^1 \binom{N}{h}\mu^h(1-\mu)^t}$
This can be expressed as a beta function:
$f(\mu|H=h,T=t) = \frac{\mu^h(1-\mu)^t}{B (h+1)(t+1)}$\\
Which using factorials can be written:
$f(\mu|H=h,T=t) = \frac{(h+t+1)!}{h!t!} \mu^h(1-\mu)^t$\\
For the first data($D_1$) the function becomes:
$ \frac{(1+1+1)!}{1!1!}\mu (1-\mu = 6\mu - 6\mu^2)$\\
For the second data($D_2$)
$\frac{(3+0+1)!}{3!1!}(1-\mu)^3 = 4\cdot (1-\mu)^3$\\
For function B:
As the $g(\mu) = -6\mu^2 + 6\mu$ the Beta function is changed as follows:\\
$f(\mu|H=h,T=t) = \frac{\binom{N}{h}\mu^h(1-\mu)^t -6\mu (1- \mu)}{\int_0^1 \binom{N}{h}\mu^h(1-\mu)^t -6\mu (1- \mu)}$\\
$f(\mu|H=h,T=t) = \frac{\mu^{h+1}(1-\mu)^{t+1}}{\int_0^1 \mu^{h+1}(1-\mu)^{t+1}}$\\
This is a Beta function with h+2 and t+2. So this will be transform the previous Beta function with factorials:
$f(\mu|H=h,T=t) = \frac{(h+1+t+1+1)!}{h+1!t+1!} \mu^{h+1}(1-\mu)^{t+1}$\\
For the $D_1$:
$\frac{5!}{2!2!} \mu^2(1-\mu)^2 = 30\mu^2(1-\mu)^2$\\
For the $D_2$:
$\frac{6!}{4!} \mu(1-\mu)^4 = 30\mu(1-\mu)^4$\\
\subsection{}
For the $\mu_{ML} = \frac{Number of heads}{Number of rounds}$
So for $D_1$: $\mu_{ML} = \frac{1}{2}$\\
For $D_2$: $\mu_{ML} = \frac{0}{3} = 0$\\
\subsection{}
To find the MAP estimate will use the function that are created from the function:
$\mu_{MAP} = \frac{h+a-1}{N+b+a-2}$\footnote{http://www.mi.fu-berlin.de/wiki/pub/ABI/Genomics12/MLvsMAP.pdf :page 18}. \\
For A: $a=1,b=1$ and $D_1$: $\frac{1+1-1}{2+1+1-2} = \frac{2}{4} = \frac{1}{2}$\\
For the $D_2$: $\frac{0+1-1}{3+1+1-2} =0$\\
For the B function $a=2,b=2$:\\
For $D_1$: $\frac{1+2-1}{2+2+2-2} = \frac{1}{2}$\\
For $D_2$: $\frac{0+2-1}{3+2+2-2} = \frac{1}{5}$\\
\subsection{}
For the full bayesian probability we know that is the same as teh Bayes' estimator.
So we will use the function for finding the full bayesian probability as $E[\mu\mid D] = \frac{H+a}{N+a+b}$\footnote{http://www.math.uah.edu/stat/point/Bayes.html}.\\
For A and $D_1$: $\frac{1+1}{2+1+1} = \frac{1}{2}$\\
For $D_2$: $\frac{1+0}{3+1+1} = \frac{1}{5}$\\
For B and $D_1$: $\frac{1+2}{2+2+2} = \frac{1}{2}$\\
For $D_2$: $\frac{0+2}{3+2+2} = \frac{2}{7}$\\
\subsection{}

